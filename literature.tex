\section{Literature Review}
% Introduction to patent:
% Patent has a lot of information about innovation, represents the inventive progress of social.
% A good source to study about change and trend in science and technology.
% General research
% Study patent citation to learn about multiple linkage between inventors, organization, firms, etc.
% Economic: relating patent counts to industry & Compustat firms
% Find all distinct inventors and disambiguate them.
% Extract and format patent XML data

\subsection{United States Patent Research}

\paragraph{US Patent Dataset.}  United States patent dataset has become widely
accessible to researchers and public since the early 1990. The rich
information included in the dataset covers all the patents filed in the US
in the last four decades. Each patent contains all the information about an
innovation, and together all of them represent the inventive progress of
technology and science --- not only in the US, but also around the
world~\footnote{In the late 1990s, around 45\% of US patents were awarded to
foreign inventors.}. More specially, each patent is highly detailed to cover
every aspect of the invention, enabling us to study the progress from
different angles. For example, information included in a patent contains  the
technological area, the inventors, the assignee,  terms of fields, types of
inventors, citation to another patents, etc. The dataset is also a good
source  for investigating the ground-breaking inventions as well as the trends in 
innovation over a particular period of time.

\paragraph{US Patent Research Analysis.} Early research of the US patent data goes
as back as the 1960s, when many researchers used patent data in economics
analysis and technology change. For example, in 1966,  Schmookler related 
industries to patent counts~\cite{Schmookler1966}, or in 1982, Griliches
addressed the problem of matching patents to Compustat
firms~\cite{Griliches1982}. The common approach of the early research is that
the rely heavily on the patent counts to come up with some indicative insight.
The simple patent count matrix does not allow them to faithfully capture more
aspects of the information included in each patent and the relation between
patents~\cite{Griliches1987}.

On other hand, researchers often model the patent dataset as
an ``inventor network", in which nodes are inventors or patents, while an edge
represents a citation between patents or a collaboration between two inventors.
Studying this network reveals multiple relations between inventors,
organizations, firms, regions, etc. In~\cite{Hall01thenber}, Hall~\etal
initiated this line of research by pointing out and addressing several issues
while analyzing citation data. They indicated that, due to the drastic change
in the rate of patenting, it is quite challenging to study the {\em received}
citation data for even a small set of patents. The problem is that to collect
all received citations of a particular patent {\em P} granted in year {\em t},
one has to visit all patents granted from year {\em t}. Hall~\etal also propose
two possible approaches to eliminate the problem, we refer the interested
reader to their paper. Using their proposed approaches, they show the main
trends in US patenting activity over three decades by different means of
measurement and across several main technology categories. Many later works
follow Hall~\etal's direction to study different problems in patent
analysis~\cite{Leskovec:2005, Hall2000, leskovec2007graph, Acs2002}.

\paragraph{Extracting and formating data in patent.}  A general technical
problem when doing patent analysis is --- how to extract and format data from
patent dataset? Typically, the data is available in Extensible Markup
Language (XML) format. However, since the patent is often objectively prepared
by individuals, there is no consistent convention in naming, data field,
etc. throughout the dataset, thus limiting potential statistical and
analytical analysis. For example, if an inventor files a patent, he can 
list his name as John Doe, or John D., or even John S. Doe, etc. Researchers when
processing the data must be aware of these inconsistency and deficiencies to
generate a precise and comprehensive results. Recent work has addressed this
problem by building a parser, which takes into account relevant information 
such as geo-location for disambiguating inventor name and / or organization~\cite{formattingpatentdata,
disambiguation, Torvik:2009}. 

 
\subsection{Shortest Path Problem} 
%
Given a graph $G = (V, E)$, we study the problem of finding the shortest path
between a source node $u \in V$ and a destination node $v \in V$. In normal
graph, the length of a path is the total of nodes along that path. However, in
weighted graph where each edge $e_i = (u, v)$ has a weight $w_i$, the length
of a path is computed as the total weight of all edges in the path. For
simplicity, we consider the normal graph as the weighted one with all $w_i=1$.

\paragraph{Shortest path from a given vertex.}  If the source node $u$ is
fixed, there are two popular algorithm to find the shortest path from $u$ to
any $v$.
	\begin{itemize}
		\item {\bf Dijkstra algorithm} works only for a graph with non-negative 
		edge weights. The general idea of Dijkstra algorithm is to find the
		nearest unvisited node to $u$, namely $k$, at every iteration. After
		that, the distances of $k$'s neighbors to $u$ are updated based on
		the distance of $k$ to $u$ and $k$ to that node. The time complexity
		of the algorithm without / with a min-priority queue are $O(|V|^2)$ and
		$O(|E|+|V|\log|V|)$ respectively (where $|V|$ is the number of nodes,
		$|E|$ is the number of edges).

		One simple trick to make Dijkstra algorithm work on graph with 
		negative edge weights is as following. Call $w_n < 0$ is the smallest negative
		weight of an edge in the graph, we add $|w_n|$ to every edge in $G$ to
		make it a non-negative edge graph $G'$. After that we run Dijkstra
		algorithm normally on $G'$ and deduct $|w_n|$ from all the edge in the
		result shortest path to get the correct answer.

        \item {\bf Bellman-ford algorithm} runs slower than Dijkstra
		algorithm, however it works with graphs with edges of arbitrary
		weight~\footnote{Note that it is not practical to find a shortest path in a
		graph with a negative circle.}. The algorithm can also detect the negative
		circle in the graph. The main insight of Bellman-ford algorithm can be
		explained in the context of dynamic programing. The key
		observation is that any shortest path from $u$ to $v$ in $G$ will have at most
		$|V|−1$ edges. Thus, we answer the question by sequentially looking for
		shortest path between $u$ and $v$ which has at most $1, 2, ..., |V|-1$ edges.
		The algorithm runs in  $O(|V|\cdot |E|)$ time, where $|V|$ and $|E|$ are the
		number of vertices and edges respectively.
	\end{itemize}

\paragraph{Shortest path between any pair of vertices.}
The problem of shortest path between any pair of vertices can be easily solved
by running $|V|$ times the above problem of shortest path from a given vertex.
The best complexity of such approach is by using Dijkstra algorithm with min-
priority queue ($O(|V|\cdot|E|+|V|^2\log|V|)$). We will discuss the other
alternatives to address the problem in this section.
	\begin{itemize}	
		\item {\bf Floyd-Warshall algorithm} runs in $O(|V|^3)$. The pseudo-code 
		is illustrated in Algorithm~\ref{alg:floyd-war}. The key idea is
		to improve the estimated distance between $u$ and $v$ by visiting all
		the nodes and see if the node is in the shortest path between $u$ and
		$v$.

	 	\item {\bf Johnson algorithm} is more efficient than Floyd-Warshall
	 	algorithm when the graph is sparse ($|E| \ll |V|^2$). The algorithm
	 	runs in $O(|V|^2\log|V| + |V||E|)$ time. Interested readers can refer for
	 	more details to~\cite{johnson-alg}.
	\end{itemize}

\begin{algorithm}
	\begin{algorithmic}[1]		
		\ForAll{$v \in V$} 
		   \State $dist[v][v] \gets 0$	\Comment{$dist$ is a $|V| \times |V|$ array of minimum distances initialized to infinity}
		\EndFor
		\ForAll{edge $(u,v) \in E$}
		   \State $dist[u][v] \gets w(u,v)$  \Comment{the weight of the edge $(u,v)$}
		\EndFor
		\ForAll{$1 \leq k \leq |V|$}
		    \ForAll{$1 \leq i \leq |V|$}
		      \ForAll{$1 \leq j \leq |V|$}
		         \If {$dist[i][j] > dist[i][k] + dist[k][j]$} 
		            \State $dist[i][j]\gets  dist[i][k] + dist[k][j]$
		        \EndIf
		      \EndFor
		    \EndFor
		\EndFor
	\end{algorithmic}
\caption{Floyd-Warshall algorithm, cited from~\cite{floyd-war-wiki}}
\label{alg:floyd-war}
\end{algorithm}

\subsection{Eigenvector centrality}
%
It measures the relative centrality of nodes by taking its neighbor nodes into
account. The scoring system is designed to differentiate between the
connections to a high-scoring node and a low-scoring node, in a sense that
connecting to the high-scoring node will make the node more central. Given a
graph $G:=(V,E)$ and let $A = (a_{v,t}\in\{0, 1\})$ be the adjacency matrix,
the formula to compute the score of a node $v$ is illustrated in
Equation~\eqref{eq:eigenvector}

\begin{equation}
	\label{eq:eigenvector}
	x_v = \frac{1}{\lambda} \sum_{t \in M(v)}x_t = \frac{1}{\lambda} \sum_{t \in G} a_{v,t}x_t
\end{equation}
where $M(v)$ is the collection of the neighbors of v and $\lambda$ is a
constant called eigenvalue~\cite{Eigenvecto-wiki}.

\paragraph{Katz centrality.}   A variant of Eigenvector centrality is Katz
centrality, which computes a node score also by taking influence of other
nodes into account. In this case, the distance from all other nodes are
considered, and the larger the distance between $u, v$, the smaller the
influence that $u$ and $v$ have on each other. Interested readers can refer 
to~\cite{katz-wiki} for more details.

\paragraph{Pagerank algorithm.}  Another variant of Eigenvector centrality is
PageRank used by Google Search~\cite{pagerank-wiki} to rank the websites in
their search results. The difference between Pagerank and Eigenvector
centrality and Katz centrality is that the connecting neighbors are taken into
account, not the connected nodes. For example, in Eigenvector centrality, all
nodes $v$ so that $a_{u, v}= 1$ will have affect on $u$'s score, while in
Pagerank, only nodes $v$ having $a_{v, u} = 1$ will do.