\section{Solution Approach}

\subsection{Inventor Network Model}

\paragraph{Graph G1: Undirected Graph.}

	\begin{itemize}
	%\squish
		\item {Vertices:} Inventors  ($I_1$, $I_2$, ...)

		\item {Edges:} Patent IDs
	\end{itemize}

There exists an edge between $I_1$ and $I_2$ if these two inventors have a
joint patent. Since two inventors can have multiple joint patents, the weight
signifies the number of joint patents

\paragraph{Graph G2: Directed Graph.}

	\begin{itemize}
	%\squish
		\item {Vertices:} Inventors ($I_1$, $I_2$, ...) or Patents ($P_1$, $P_2$, ...)

		\item {Edges:} References or Inventions
	\end{itemize}

There exists an `invention' edge between $I_1$ and $P_1$ if $I_1$ is an inventor of $P_1$.
There exists a directed `reference' edge from $P_1$ to $P_2$ if $P_1$ references $P_2$

\subsection{Definitions \& Insights}

	\begin{itemize}
	%\squish
		\item {\em Collaborative distance.} For any two inventors, we define
		the collaborative distance as the length of the shortest path in
		Graph G1.

		\item {\em Invention impact.} For a given inventor, we define his /
		her invention impact as the eigenvalue centrality for this inventor
		in Graph G2.
	\end{itemize}

% \subsection{Insight}

In our model, the impact of an inventor is measured by the number of patents
he / she invents (quantity) as well as the number of references for each of
these patents (quality). Thus, intuitively, the more is the eigenvalue
centrality measure, the inventor is more `impactful'. Similarly, we use
collaborative distance to measure the `connectedness' between two inventors.
Thus, intuitively, the smaller the collaborative distance, the closer they are
to each other in terms of connection in the graph.  In a nutshell, the answers
to our questions framed in our problem description follow from these insights:

\paragraph{Answer 1.} Check the correlation between collaborative distance and invention impact. 

\paragraph{Answer 2.} Check the correlation between collaborative distance and organization.

\paragraph{Answer 3.} Check the correlation between invention impact and organization.


\subsection{Analysis Methodology}

	\begin{itemize}
	%\squish
		\item Gephi and / or IGraph: Graph analysis and metric calculation~\cite{gephi, igraph}.

		\item GraphViz, Gnuplot: Data visualization and result plotting~\cite{graphviz, gnuplot}.

		\item Standard libraries and custom Python Scripts: For data extraction~\cite{python}.
	\end{itemize}