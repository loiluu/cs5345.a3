\section{Solution Approach}
\label{sec:sol}
With our problem definition narrowed down to three sub-questions, we build a social graph model to address these research questions. 
After we have the social graph, we select appropriate network metrics to measure the aspects of innovation.
With these metrics, we can answer the research questions as well as provide empirical evidence to verify our rankings.

\subsection{Inventor Network Model}
\label{sec:model}
Our social graph models two aspects of the inventor network:  `Which inventors
and organizations have joint patents?' and `How many patents cite a certain
patent?'. These two networks abstracts out just enough information to address
our three sub-problems. We describe our graph models in detail below:

\paragraph{Graph G1: Co-inventorship Graph}

	\begin{itemize}
	\squish
		\item {Vertices:} Inventors  ($I_1$, $I_2$, ...)

		\item {Edges:} Signify co-inventorship on one / more patents
	\end{itemize}

There exists an edge between $I_1$ and $I_2$ if these two inventors have a
joint patent. Since two inventors can have multiple joint patents, the weight
signifies the number of joint patents

\paragraph{Graph G2: Citation Graph.}

	\begin{itemize}
	\squish
		\item {Vertices:} Inventors ($I_1$, $I_2$, ...) or Patents ($P_1$, $P_2$, ...)

		\item {Edges:} Citations or Inventions
	\end{itemize}

There exists an `invention' edge between $I_1$ and $P_1$ if $I_1$ is an inventor of $P_1$.
There exists a directed `citation' edge from $P_1$ to $P_2$ if $P_1$ cites $P_2$.

\subsection{Definitions \& Insights}

	\begin{itemize}
	\squish
		\item {\em Collaborative distance.} For any two inventors, we define
		the collaborative distance as the length of the shortest path in
		Graph G1.

		\item {\em Invention impact.} For a given inventor, we define his /
		her invention impact as the eigenvalue centrality for this inventor
		in Graph G2.
	\end{itemize}

% \subsection{Insight}

In our model, the impact of an inventor is measured by the number of patents
he / she invents (quantity) as well as the number of references for each of
these patents (quality). Thus, intuitively, the more is the eigenvalue
centrality measure, the inventor is more `impactful'. Similarly, we use
collaborative distance to measure the `connectedness' between two inventors.
Thus, intuitively, the smaller the collaborative distance, the closer they are
to each other in terms of connection in the graph.  In a nutshell, the answers
to our questions framed in our problem description follow from these insights:

\paragraph{Answer 1.} Check the correlation between collaborative distance and invention impact. 

\paragraph{Answer 2.} Check the correlation between collaborative distance and organization.

\paragraph{Answer 3.} Check the correlation between invention impact and organization.
