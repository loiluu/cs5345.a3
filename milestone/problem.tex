\section{Problem Description}

\paragraph{Motivation.}

A variety of surveys, reports, and yearly analysis efforts try to ranking
organizations  based on their innovation. The underlying reason is to measure
which companies and / or products are influencing the market the most.  These
studies consider factors such as the monthly sales, returns, innovation
premiums, royalty earned, the global impact of the innovation, etc. Although,
its crucial to  consider the impact of innovation for ranking the
organizations, the metric for such measurement  varies largely across all
studies. The consequence is that there is a significant dissonance in the
rankings of all the surveys.  This motivates the need for a standard technique
to compare the innovations on fair grounds and in a holistic way. 

To this end, in this work we argue that patents as one of the main factors for
measuring innovation of an organization and / or an individual. It is true
that patents alone are an insufficient measure to compare the overall
innovation since research papers, consumer products, services are also a major
way of displaying innovation. However, such a measure can be considered as one
of the factors for weighted rankings. Thus, given the patent history of
various organizations for past about four decades,  how should one go about in
using this information to measure innovation --- quantifying an abstract
concept such as innovation is not straightforward as it may seem.  

\subsection{Objectives}

To address this open challenge, we narrow down our efforts towards tackling
three main  aspects of innovation. Firstly, should the metric purely consider
the quantity of patents filed per unit time (for e.g., month, year, or a
decade) or should it be based on the quality of an invention (for e.g., a
groundbreaking invention such as RSA cryptosystem vs. a insignificant patent
of shapes of chocolates)? Secondly, having a handful of prolific inventors in
an organization may be sufficient to churn patents rapidly, but does it mean
that the organization as a whole id innovative? Lastly, how is the dynamics
between co-inventors, do they tend to collaborate more often with other
impactful inventors, does their involvement affect the patent grant in any
way? This helps to throw light on the evolution of innovation within an
organization, thus giving insights about the future prospects of innovation
for a given organization.

For a systematic study of the above factors, we confine our problem to
answering the following research questions:

\paragraph{Question 1.} Is a inventor more impactful because he is closely
connected to other impactful inventors in the patent social network? 
%(c)(eigen erdos)

\paragraph{Question  2.} Are inventors closely connected to an impactful
inventor in their own organization or other organizations? %(a) (erdos org)

\paragraph{Question  3.} Has the organization made impactful inventions in
recent past? %(b) (eigen org)


\subsection{Assumptions}

We use the USPTO patent grant full text dataset from January 2005 to February
2015.  The dataset includes patent number, series code and application number,
type of patent, filing date, title, issue date, inventor information, assignee
name at time of issue, foreign priority information, related US patent
documents, classification information, US and foreign references, attorney,
agent or firm/legal representative, Patent Cooperation Treaty (PCT)
information, abstract, specification, and claims for each patent. We consider
only the following fields from the dataset for our study: patent number, issue
date, inventor name, inventor address, assignee name at time of issue, US
references.

	\begin{itemize}
	%\squish
		\item We consider only a subset of the entire patents that were granted till
		date. Specifically, we study patents granted from year 2005 to 2015. This
		assumption helps us to reduce the dataset size, allows to study the current
		trend in innovation and makes the pre-processing easier due to the
		consistency in data format (all the archives are in XML from year 2005
		onwards). Hence, we ignore all the references to patents before 2005.

		\item We also ignore the references to Non-US patents and other sources. item
		To identify whether similar names correspond to a single inventor (e.g., John
		Doe and John S. Doe), we consider if combination of first name, last name,
		state and country is unique, and if so, group all such names under a single
		inventor.  item Patents with Missing organization name  are assigned to
		`Individual patents'. item If the two nodes for collaborative distance belong
		to two disconnected graphs, then we assume the distance to be infinite.
	\end{itemize}


\subsection{Contributions}
