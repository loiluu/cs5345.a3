\section{Problem Description}

%\subsection{Motivation}

A variety of surveys, reports, and yearly analysis efforts try to ranking
organizations  based on their innovation. The underlying reason is to measure
which companies and / or products are influencing the market the most.  These
studies consider factors such as the monthly sales, returns, innovation
premiums, royalty earned, the global impact of the innovation, etc. Although,
its crucial to  consider the impact of innovation for ranking the
organizations, the metric for such measurement  varies largely across all
studies. The consequence is that there is a significant dissonance in the
rankings of all the surveys.  This motivates the need for a standard technique
to compare the innovations on fair grounds and in a holistic way. 

To this end, in this work we argue that patents as one of the main factors for
measuring innovation of an organization and / or an individual. It is true
that patents alone are an insufficient measure to compare the overall
innovation since research papers, consumer products, services are also a major
way of displaying innovation. However, such a measure can be considered as one
of the factors for weighted rankings. Thus, given the patent history of
various organizations for past about four decades,  how should one go about in
using this information to measure innovation --- quantifying an abstract
concept such as innovation is not straightforward as it may seem.  

\subsection{Objectives}

To address this open challenge, we narrow down our efforts towards tackling
three main  aspects of innovation. Firstly, should the metric purely consider
the quantity of patents filed per unit time (for e.g., month, year, or a
decade) or should it be based on the quality of an invention (for e.g., a
groundbreaking invention such as RSA cryptosystem vs. a insignificant patent
of shapes of chocolates)? Secondly, having a handful of prolific inventors in
an organization may be sufficient to churn patents rapidly, but does it mean
that the organization as a whole id innovative? Lastly, how is the dynamics
between co-inventors, do they tend to collaborate more often with other
impactful inventors, does their involvement affect the patent grant in any
way? This helps to throw light on the evolution of innovation within an
organization, thus giving insights about the future prospects of innovation
for a given organization.

For a systematic study of the above factors, we confine our problem to
answering the following research questions:

\paragraph{Question 1.} Is a inventor more impactful because he is closely
connected to other impactful inventors in the patent social network? 
%(c)(eigen erdos)

\paragraph{Question  2.} Are inventors closely connected to an impactful
inventor in their own organization or other organizations? %(a) (erdos org)

\paragraph{Question  3.} Has the organization made impactful inventions in
recent past? %(b) (eigen org)


\subsection{Assumptions}
\label{sec:assumptions}


	\begin{itemize}
	\squish
		
		\item We consider only a subset of the entire patents that were granted till
		date. Specifically, we study patents granted from year 1975 to 2013. There are
		two main reasons for selecting this time frame: (a) Patent data set before
		1975 is only available in OCR format, which is difficult to parse, and
		increases the efforts for preprocessing. (b) The disambiguation algorithm
		requires training models, which are available from previous research. However,
		the models and parameters are available only till year 2013, when the study
		was conducted. To maintain the precision in disambiguation and avoid re-
		running the classification on the whole database again, we limit our study to
		patent till 2013.

		\item We ignore the references to Non-US patents and other sources, since it
		is difficult to track all such sources. Please refer to
		Section~\ref{sec:changes} for the rationale for this assumption.
	\end{itemize}


\subsection{Contributions}

To summarize, we make the following contributions:

\begin{itemize} 
\squish
\item {Study the inventor network.} We study various properties of the inventor network in the US grant patent dataset for past four decades.  
\item {Metrics for measuring innovation.} We device a fair metric to measure the innovativeness of an inventor and corresponding organization. This allows us to rank various organizations based on their innovations. 
\item {Soundness of metrics.} We verify if our metrics are sound to qualitatively and quantitatively measure the innovation impact of an organization by comparing our ranking results to public lists.
\end{itemize}


% References:
% http://www.fastcompany.com/section/most-innovative-companies-2015
% http://www.forbes.com/innovative-companies/
% https://www.bcgperspectives.com/most_innovative_companies
% http://top100innovators.com/
% http://www.brw.com.au/lists/50-most-innovative-companies/2014/