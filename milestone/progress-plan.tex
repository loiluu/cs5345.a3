\section{Project Progress and Plan}
\label{sec:changes}
After a few rounds of discussions and actually working with the dataset and
the allied tools, we amended our  size of data set and  assumptions. Here, we
will also clarify a few aspects of our proposal in more detail.

\paragraph{Challenges and changes to the proposal.}

	\begin{itemize}
		\squish
		\item {\em Dataset Size.} After reconsidering the comments on the proposal, and
		studying the available tools for extracting the patent data, we realized that
		we can extend our data set to consider patents from 1975 onwards. Further,
		for precise disambiguation, we only consider patents uptill 2013 (See
		Section~\ref{sec:assumption} for details.)

		\item {\em Organization Disambiguation.} Since organizations are the focus of
		our study, disambiguating the names seems to be crucial, and hence we will
		now run a disambiguation step for the organization step before constructing
		our citation network.

		\item {\em Citations.} One of the suggestions was to consider previous patents
		only for citation information. However, we are only concerned about `Who
		cited this patent?', to decide if it is impactful. In this case, we do not
		need data before 1975, since all the patents after 1975 will be cited only by
		patents after 1975 (for e.g., a patent granted in 2007 will be cited only in
		future i.e., from 2007 onwards.), thus we do not miss any citation data. As
		for the non-US patents and other sources, it is not trivial to track `Who all
		cited this patent outside the patent network?', and this we do not consider
		the non-US and other sources which cite the patents in our dataset.

	\end{itemize}


\subsection{Progress}
\label{sec:progress}
So far, we have completed the data extraction and first phase of our
evaluation, which is based on the co-author network. 

%discuss why we swapped 2 and 3
	\begin{itemize}
		\squish
		\item {\em Data Processing:} We extract the data from raw XML to SQL and
		GraphML format for years 2005 to 2013, for the rest of the data, we relied on
		the already available processed data from Harvard.

		\item {\em Inventor Disambiguation:} We used the previously proposed pre-
		processing and disambiguation algorithm to generate a list of unique
		inventors.

		\item {\em Evaluation Phase I - Collaborative Distance:} We calculated the 
		shortest path of each inventor w.r.t to Thomas Edison. We used three
		algorithms viz. Djikstra, Bellman-Ford, Johnson from IGraph to do this. We
		also analyzed their scalability with respect to the invention network graph
		and compared their performance.

		\item {\em Social Network Behavior:} Computed the distribution of the 
		collaborative distance w.r.t. Thomas Edison for our network, and also studied
		if the overall network follows power law for various popularity measures.

	\end{itemize}

\subsection{Plan for next activities}
 
Next goal is the second phase of evaluation for the citation network and the
final analysis based on all the metric computations.

	\begin{itemize}
	\squish

		\item {\em Organization / Assignee Disambiguation:} For the next phase of
		evaluation, we will run a simple pre-processing algorithm for disambiguating
		organization names. For example, IBM Corp. vs. International Business Machines
		Corporation are considered different organizations currently. We will use a
		similar algorithm as for inventor disambiguation, to merge multiple assignees
		with such names into a single entity.

		\item {\em Evaluation Phase II - Eigenvalue centrality:} Use the Newman's
		leading eigenvector algorithm in IGraph to calculate eigenvalue centrality
		for each unique inventor with respect to patents that cite his / her patent.
		% ~\footnote{We skipped this activity for now, because it relies on the citation graph. Instead we did the next activity which uses the co-inventorship graph that we generated in this milestone.}.

		\item {\em Data analysis and comparison:} Plot the graphs for the above
		generated data and analyze the relationship between the above two
		measurements. Compare our ranking results for the most innovative organization
		with publicly available list such as Forbes list, to check if our ranking
		metric coincides with the results of other metrics.

		\item{\em Considering time as a factor:} Optionally, we plan to study the evolution 
		of the citation and the co-authorship network over a period of time, and if forming a 
		link leads to increase / decrease in our measures. We have retained the time 
		information in our data so far, but we may not be able to finish this task 
		completely given the time constraints.

	\end{itemize}

% \begin{table}[H]
% %\footnotesize
% \centering
% \begin{tabular}{| l | l | l |}
% \hline
% {Date} & {Activity} & Progress \\
% \hline
% \hline
% 24th March & Data Extraction and Author Disambiguation & \tick \\
% 31st March & Calculate Eigenvalue centrality for all nodes & - \\
% 7th April & Calculate shortest path between nodes - Run 3 algorithms & \tick \\
% 14th April &  Comparison charts and analysis & - \\
% 17th April & Final Report Submission & - \\
% \hline
% \end{tabular}
% \label{tbl:milestone}
% % \vspace{-0.5cm}
% \end{table}