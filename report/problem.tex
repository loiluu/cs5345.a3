\section{Problem Description}
\label{sec:prob}
%\subsection{Motivation}

A variety of surveys, reports, and yearly analysis efforts to rank
organizations are based on their innovation. The underlying motive is to measure
which companies and / or products are influencing the market the most.  These
studies consider factors such as the monthly sales, returns, innovation
premiums, royalty earned, the global impact of the innovation, etc. Although,
its crucial to consider the impact of innovation for ranking the
organizations, the metric for such measurement  varies largely across all these 
studies. The consequence is that there is a significant dissonance in the
rankings of all the surveys.  This motivates the need for a standard technique
to compare the innovations on fair grounds and in a holistic way. 

To this end, in this work we argue that patents are one of the main factors for
measuring innovation of an organization and / or an individual. It is true
that patents alone are an insufficient measure to compare the overall
innovation since research papers, consumer products, services are also a major
way of displaying innovation. However, such a measure can be considered as one
of the factors for weighted rankings. Thus, the question is -- given the patent history of
various organizations for past about four decades,  how should one go about in
using this information to measure innovation? Quantifying an abstract
concept such as innovation may not be intuitive and straightforward as it may seem.  

\subsection{Objectives}

To address this challenge, we narrow down our efforts towards tackling
three main  aspects of innovation. Firstly, should the metric purely consider
the quantity of patents filed per unit time (for e.g., month, year, or a
decade) or should it be based on the quality of an invention (for e.g., a
groundbreaking invention such as RSA cryptosystem vs. a insignificant patent
of shapes of chocolates)? Secondly, having a handful of prolific inventors in
an organization may be sufficient to churn patents rapidly, but does it mean
that the organization as a whole is innovative? Lastly, how is the dynamics
between co-inventors, do they tend to collaborate more often with other
impactful inventors, does their involvement affect the frequency and quality 
of patent grant in any way? This helps to throw light on the evolution of 
innovation within an organization, thus giving insights about the future 
prospects of innovation for a given organization.

For a systematic study of the above factors, we confine our problem to
answering the following research questions:

\paragraph{Question 1.} Is an inventor more impactful because he is closely
connected to other impactful inventors in the patent social network? 
%(c)(eigen erdos)

\paragraph{Question  2.} Are inventors closely connected to an impactful
inventor in their own organization or other organizations? %(a) (erdos org)

\paragraph{Question  3.} Which organizations are leading the innovation industry via impactful inventions in last four decades? %(b) (eigen org)


\subsection{Assumptions}
\label{sec:assumptions}
	\begin{itemize}
	\squish
		\item We consider only a subset of the entire patents that were granted till
		date. Specifically, we study patents granted from year 1975 to 2013. There are
		two main reasons for selecting this time frame: (a) Patent data set before
		1975 is only available in OCR format, which is difficult to parse, and
		increases the efforts for preprocessing. (b) The disambiguation
		algorithm~\cite{disambiguation}  requires training models and custom
		parameters, which are available from previous research. However, the models
		and parameters are available only till year 2013, when the study was
		conducted. To maintain the precision in disambiguation and avoid re- running
		the classification on the whole database, we limit our study to patents
		granted till 2013.
		\item We ignore the references to Non-US patents and other sources, since it
		is difficult to track all such sources.  We are only concerned about `Who
		cited this patent?', to decide if it is impactful. In this case, we do not
		need data before 1975, since all the patents after 1975 will be cited only by
		patents after 1975 (for e.g., a patent granted in 2007 will be cited only in
		future i.e., from 2007 onwards.), thus we do not miss any citation data. As
		for the non-US patents and other sources, it is not trivial to track `Who all
		cited this patent outside the patent network?', and this we do not consider
		the non-US and other sources which cite the patents in our dataset.
	\end{itemize}

\subsection{Contributions}

To summarize, we make the following contributions:

	\begin{itemize} 
	\squish
		\item {\em Study the inventor network.} We study various properties of the inventor network in the US grant patent dataset for past four decades.  
		\item {\em Metrics for measuring innovation.} We device a fair metric to measure the innovativeness and impact of an inventor and corresponding organization. This allows us to rank various organizations based on their innovations alone. 
		\item {\em Soundness of metrics.} We verify if our metrics are sound to measure the innovation impact of an organization. Specifically, we compare our ranking results to public lists such as Reuters and 24/7 Wall Street.~\cite{top100, 247wallst}.
	\end{itemize}